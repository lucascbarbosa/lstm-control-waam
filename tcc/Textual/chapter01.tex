\chapter{Introdução}
No contexto da Indústria 4.0, o setor fabril testemunhou grandes avanços no desenvolvimento da Manufatura Aditiva (MA), que permite a fabricação de peças de geometrias complexas com grandes liberdades de design e reduzido gasto de material e tempo (40-60\% do tempo de fabricação e 15-20\% do tempo de pós processamento). A MA tem se estabelecido como uma das mais promissoras tecnologias na indústria moderna. Entre suas diversas modalidades, destaca-se a Manufatura Aditiva por Arco e Arame (WAAM - Wire Arc Additive Manufacturing), uma técnica que vem ganhando crescente atenção e aplicação em diversos setores industriais, através do uso de manipuladores robóticos modernos. Um sistema básico de MA consiste na combinação de um sistema de movimento, uma fonte de calor e um material usado para a formar a peça fabricada. No caso de WAAM, o manipulador é encarregado da trajetória, a fonte é responsável por fundir o arame, o material usado para a fabricação das peças, geralmente ligas de Titânio, Alumínio ou Aço.\cite{williams2016wire, wu2018review}. 

A utilização de técnicas de MA como a WAAM se diferem das demais na capacidade de produzir peças de grandes dimensões como peças de aviões e foguetes, que possuem uma geometria não trivial. Porém, apresentam-se diversos riscos ao processo, devido a grande quantidade de calor absorvida pela peça, levando a deformações causadas pela tensão residual acumulada e um acabamento insatisfatório, como camadas de cordões sem uma uniformidade geométrica (altura e largura constantes) \cite{ding2015wire}.

A inspeção é uma etapa fundamental do processo, ajudando na detecção de defeitos e aumentando a confiança no produto final. Porém, além de ser uma tarefa não trivial, a avaliação de qualidade do produto não é suficiente para dar conta de todos as possíveis falhas de fabricação. Com isso, torna-se necessário o desenvolvimento de um sistema que seja capaz de controlar de forma eficiente as características geométricas da peça fabricada por WAAM. Este trabalho foca no desenvolvimento de uma Rede Neural (RN) capaz de identificar a dinâmica não linear e complexa do processo de impressão via WAAM, estimando as características geométricas (altura e largura) do cordão de solda a partir das séries temporais de parâmetros de solda da fabricação e da configuração do robô (Corrente e Velocidade de Alimentação de Arame (\textit{Wire Feed Speed} - WFS), utilizando a arquitetura das redes de Memória de Curto-Longo Prazo (\textit{Long Short Term Memory} - LSTM). Posteriormente, essa RN é utilizada como modelo dentro de um controlador MPC, com o objetivo de criar um sistema de controle inteligente.

\section{Motivação}
Com o desenvolvimento e modernização cada vez mais acelerada dos processos fabris, sistemas de MA como WAAM serão protagonistas na fabricação de peças complexas para os setores da aviação e aeroespacial, por exemplo. Com isso, é de extrema importância e urgência que sejam desenvolvidos sistemas inteligentes para controle da fabricação das peças através dessa técnica, com o intuito de melhorar sua qualidade. Para isso, é necessário desenvolver modelos capazes de identificar a dinâmica não linear do processo, mapeando os parâmetros de solda às características geométricas resultantes, para que seja possível fechar a malha de controle do sistema inteligente de WAAM. Portanto, o desenvolvimento da RN abordada neste trabalho é relevante já que não há um modelo consolidado na indústria capaz de realizar esta tarefa nos dias de hoje.

\section{Objetivos}
O objetivo deste trabalho é criar uma rede LSTM capaz de estimar os parâmetros geométricas de cordão único a partir da série temporal dos parâmetros de soldagem, modelando a dinâmica não linear do processo GMAW. Ademais, a rede será utilizada juntamente com um otimizador quadrático como modelo dentro uma malha de controle preditivo, criando assim um sistema de inteligente de controle da geometria da peça.