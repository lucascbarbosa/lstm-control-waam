\begin{abstract}
Apresenta-se neste trabalho uma aplicação de Rede Neural capaz de estimar em tempo real as características geométricas (largura e altura) de cordões simples fabricados através da Manufatura Aditivo por Arco de Arame (WAAM). 

A área da Manufatura Aditiva (MA), vêm se tornando cada vez mais estudada e desenvolvida no contexto da Indústria 4.0 por conta da sua capacidade de produzir peças com geometrias complexas, reduzido uso de material, e propriedades mecânicas superiores (os chamados metamateriais). 

No entanto, no contexto de WAAM as dinâmicas físicas do processo ainda são pouco conhecidas, devido a alta complexidade de modelagem, fazendo com que as peças não apresentem conformação e uniformidade geométrica. Para contornar problema são utilizados modelos aproximados que não necessariamente possuem uma boa razão performance - complexidade.

Com isso, alguns estudos utilizam modelos de \textit{Deep Learning} (DL), as chamadas Rede Neurais (RN), para estimar as características geométricas do cordão (altura e largura) baseado nos parâmetros de solda (como corrente, tensão e velocidade de alimentação do arame). A arquitetura utilizada foi a \textit{Long-Short Term Memory} (LSTM), que foi acoplada a uma malha de fechada de controle funcionando como um controlador.
\end{abstract}

